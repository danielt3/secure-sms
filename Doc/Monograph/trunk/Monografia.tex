\documentclass[12pt]{article}

\usepackage{sbc-template}
\usepackage{amsmath,amssymb}
\usepackage{txfonts}
\usepackage{graphicx,url,ifthen}
\usepackage[latin1]{inputenc}
\usepackage{textcomp}
%\usepackage[portuges]{babel}
\usepackage[brazil]{babel}

\newboolean{Anonymous}
%\setboolean{Anonymous}{false}
\setboolean{Anonymous}{false}

\newcommand{\emailx}[1]{\footnotesize\texttt{#1}}

\newcommand{\F}{\mathbb{F}}
\newcommand{\G}{\mathbb{G}}
\newcommand{\ID}{\mathsf{ID}}
\newcommand{\Z}{\mathbb{Z}}

\newcommand{\LComment}[1]{\(\triangleright\) #1}

\newtheorem{definition}{Definition}

\sloppy

\title{Constru��o de um Sistema de SMS Seguro}

\ifthenelse{\boolean{Anonymous}}
{%
\author{(Anonimizado para submiss�o)\inst{1}}
\address{(Anonimizado para submiss�o)}
} %Anonymous (then)
{%
\author{
Eduardo de Souza Cruz\inst{1}, Geovandro C. F. Pereira\inst{1},\\
Rodrigo Rodrigues da Silva\inst{1}, Paulo S. L. M. Barreto\inst{1}\thanks{Orientador do trabalho. Bolsista de Produtividade em Pesquisa CNPq, processo 312005/2006-7.}.
}
\address{
    Departamento de Engenharia de Computa��o e Sistemas Digitais,\\
    Escola Polit�cnica, Universidade de S�o Paulo, Brasil.\\
    \emailx{\{eduardo.cruz,geovandro.pereira,rodrigo.silva1\}@poli.usp.br}, \emailx{pbarreto@larc.usp.br}
}
} %Anonymous (else)

\begin{document}

\pagestyle{empty}

\maketitle

\begin{resumo}
TODO TODO TODO TODO TODO TODO TODO TODO 
\end{resumo}

\section{Resumos}
\section{Introdu��o}
\subsection{Cen�rio}
\subsection{Objetivos}
\subsection{Metodologia}
\subsection{M�tricas}
\section{Discuss�o}
\subsection{Preliminares te�ricas}% (assunto geral: curvas el�pticas, protocolos, conceitos, etc)
\subsection{Alternativas existentes}% (RSA, coisas mobile)
\subsection{Revis�o literatura}
\section{Mais coisas aqui no meio... descrever o bdcps, o projeto, etc}
\section{Modelos}%, justificar o sistema
\section{Resultados}% (tabelas, etc, os objetivos foram alcan�ados? Testes, benchmarks)
\section{Conclus�o}
\subsection{Analisar resultados}
\subsection{Poss�veis desenvolvimentos futuros}% (ordenar artigos)
\section{Refer�ncias}% (falar q gerou artigo e men��o honrosa)

\section{Introdu��o}
\section{Introdu��o}


\bibliographystyle{sbc}
\bibliography{Monografia}

\end{document}
