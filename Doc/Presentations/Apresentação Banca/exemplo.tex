% Layout Options
%
% [slidestop] puts frame titles & contents on the top left corner (default = [slidescentered])
% [compress] makes all navigation bars as small as possible (bad for multi-line navigation bars)
% [red] changes navigation bars and title to reddish color (default = blue; others = red,brown,blackandwhite)
%
% Font options 
% 
% [mathserif] use serif fonts for representing formulas instead of sans serif (default = mathsans)
%
% Note options
% [notes] adds notes to PDF screen
% [notesonly] make only notes
\documentclass[notes,blue,mathserif]{beamer}
%\documentclass[slidestop,blue,mathserif]{beamer}

\usepackage{graphicx}
\usepackage{fancybox}

% This package enables us to use special letters (with accents, cedillas, etc).
% You can discard this command when the presentation is in English.
\usepackage[latin1]{inputenc}


% in \begin{document} .. \end{document}
%\mode{\usepackage{fullpage}}

% in \begin{frame} .. \end{frame}

%\usetheme{Warsaw}
\usetheme{Berlin}
%\usetheme{Malmoe}
%\usetheme{Antibes}
%\usetheme{Berkeley}
%\usetheme{Singapore}
%\usetheme{Szeged}

% The default font theme installs a sans serif for all text of the presentation
\usefonttheme{default}

% Using default block style
%\setbeamertemplate{blocks}[default]

% Background colors
% solid style
%\beamertemplatesolidbackgroundcolor{gray}
% gradient style
%\beamertemplateshadingbackground{blue!5}{yellow!10}
%\beamertemplateshadingbackground{blue!10}{yellow!5}
\beamertemplateshadingbackground{yellow!5}{blue!10}
% grid style
%\beamertemplategridbackground[0.5in]


\title{Title of Presentation}
\author{Name of Author}
\institute{Institute of Chemistry, Academia Sinica}
%\date{\today}
\date{December 12, 2007}