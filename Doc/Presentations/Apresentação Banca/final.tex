
\documentclass[notes,blue,mathserif]{beamer}

\usepackage[latin1]{inputenc}
\usepackage[brazil]{babel}
\usepackage{graphicx}
\usepackage{fancybox}
\usepackage{hyperref}


\usetheme{Boadilla}

% The default font theme installs a sans serif for all text of the presentation
%\usefonttheme{default}

% Using default block style
%\setbeamertemplate{blocks}[default]

% Background colors
% solid style
%\beamertemplatesolidbackgroundcolor{gray}
% gradient style
%\beamertemplateshadingbackground{blue!5}{yellow!10}
%\beamertemplateshadingbackground{blue!10}{yellow!5}
\beamertemplateshadingbackground{yellow!15}{blue!5}
%\beamertemplateshadingbackground{yellow!5}{blue!10}
% grid style
\beamertemplategridbackground[0.8in]

%Código para exibir layout antes de cada seção
%\AtBeginSubsection[]
\AtBeginSection[]
{
  \begin{frame}<beamer>
    \frametitle{Agenda}
%    \tableofcontents[currentsection,currentsubsection]
     \tableofcontents[currentsection]
  \end{frame}
}


\title[SMS Seguro]{Sistema de SMS Seguro\\PCS2050 - Projeto de Formatura II \\ Apresenta\c{c}\~{a}o Final}

\author[Cruz, Pereira, Silva]{Eduardo de Souza Cruz\\
					Geovandro Carlos Crepaldi Firmino Pereira\\
					Rodrigo Rodrigues da Silva\\ 
					Orientador: Prof. Dr. Paulo S. L. M. Barreto}
\institute[PCS/EPUSP]{Departamento de Engenharia de Computa\c{c}\~{a}o e Sistemas Digitais \\ Escola Polit\'{e}cnica da Universidade de S\~{a}o Paulo}
\date{S\~{a}o Paulo, 09/12/2008}

\begin{document}

\begin{frame}
\titlepage
\end{frame}


\section{Introdu\c{c}\~{a}o}

\subsection{Motiva\c{c}\~{a}o}

\begin{frame}{Motiva\c{c}\~{a}o}
Motivação
\begin{itemize}
\item SMS armazenado em aberto nas integradoras\/Operadoras
\item Algoritmo A5 quebrado da rede GSM
\item Poucas solu��es no mercado
\item Possibilidade de transformar em produto
\end{itemize}
\end{frame}

\subsection{Objetivos}
\begin{frame}{Objetivos}
Objetivos
\end{frame}

\subsection{Cen\'{a}rio}
\begin{frame}{Cen\'{a}rio}
Cenário
 \begin{itemize}[<+->]
 \item Recursos Limitados
		\begin{itemize}
		\item Processamento
	  \item mem�ria
	  \item Largura de banda
	  \end{itemize}
\end{frame}

\section{Necessidade}

\subsection{Aplica\c{c}\~{o}es Potenciais}

\begin{frame}{Aplica\c{c}\~{o}es Potenciais}

\end{frame}

\subsection{Servi\c{c}os de Seguran\c{c}a}

\begin{frame}{Servi\c{c}os de Seguran\c{c}a}
\begin{itemize}[<+->]
\item Confidencialidade
\item Integridade
\item Autenticidade
\item Irretratabilidade
\end{itemize}
\end{frame}

\subsection{M\'{e}tricas e Requisitos}
\begin{frame}{M\'{e}tricas e Requisitos}
\begin{itemize}[<+->]
\item Tempo de espera
\item Espaço útil da mensagem
\item Tamanho da chave
\item \textit{Overhead} do protocolo
\end{itemize}
\end{frame}
\note[itemize]{\item Notes content}

\section{Solu\c{c}\~{a}o}

\subsection{Especifica\c{c}\~{a}o}
\begin{frame}{Especifica\c{c}\~{a}o}
Especificacao
\end{frame}

\subsection{Arquitetura}
\begin{frame}{Arquitetura}
Arquitetura
\end{frame}

\subsection{Implementa\c{c}\~{a}o}
\begin{frame}{Implementa\c{c}\~{a}o}
	\begin{itemize}[<+->]
	\item Curvas El�pticas 
	\item	Chaves menores
	\item Criptografia auto-certificada
	\item Criptografia baseada em Identidades
	\itemGerenciamento de chaves simplificado
	\itemIrretratabilidade
\end{frame}

\subsection{Resultados}
\begin{frame}{Resultados}
res
\end{frame}


\section{Conclus\~{a}o}
\begin{frame}{Conclus\~{a}o}
concl
\end{frame}

\subsection{Aplica\c{c}\~{a}o}
\begin{frame}
aplic
\end{frame}

\subsection{Pesquisa}
\begin{frame}
pesq
\end{frame}

\subsection{Desenvolvimentos futuros}
\begin{frame}{Desenvolvimentos futuros}
fut
\end{frame}

\begin{frame}{Refer\^{e}ncias}
fut
\end{frame}

\begin{frame}{Contato}

	\begin{exampleblock}{Site do projeto}
		\begin{center}
		\begin{large}

		\textbf{http://secure-sms.googlecode.com}

		\end{large}
		\end{center}
	\end{exampleblock}

	\begin{exampleblock}{E-mail}
		\begin{center}
		\begin{large}

		\textbf{secure-sms@googlegroups.com}

		\end{large}
		\end{center}
	\end{exampleblock}

\end{frame}






\begin{frame}{Agenda}
This is a short introduction to Beamer class.
\end{frame}

\begin{frame}{Motiva\c{c}\~{a}o}
\begin{itemize}
\pause \item Beamer is a wonderful class
\pause \item One can make animations
\pause \item One uses the \textbf{pause} command, for example
\pause \item in order to bring in important ideas
\end{itemize}
This is a short introduction to Beamer class.
\end{frame}

\begin{frame}{Introduction}
\begin{itemize}
\item<2-> appears from slide 2 on
\item<2-4> appears from slide 2 to slide 4
\item<4> appears on slide 4
\item<3-> appears from slide 3 on
\item<5-> 5
\end{itemize}
\end{frame}

\begin{frame}{Conclus\~{a}o}
\begin{itemize}[<+->]
\item L
\item A
\item T
\item E
\item X
\end{itemize}
\end{frame}

\begin{frame}

\begin{itemize}
\item Language used by Beamer: L\uncover<2->{A}TEX
\item Language used by Beamer: L\only<2->{A}TEX
\end{itemize}

\end{frame}

\begin{frame}

\uncover<2->
{appear from slide 2 on\\}
\uncover<3-4>
{appears from 3 to slide 4\\}
\uncover<4>{appears on slide 4\\}
\uncover<3->{appears from slide 3 on\\}

\end{frame}

\begin{frame}
\begin{block}{Block title}
This is a block in blue
\end{block}

\begin{alertblock}{Alert-block title}
This is a block in red
\end{alertblock}

\begin{exampleblock}{Example-block title}
This is a block in green
\end{exampleblock}
\end{frame}

\end{document}